\documentclass[a4paper,11pt,oneside, table]{article}
\usepackage[margin=0.5in]{geometry}
\usepackage{setspace}
\usepackage{imakeidx}
\usepackage{float}
\usepackage{graphicx}
\usepackage{pdfpages}
\usepackage{csquotes}
\usepackage{caption}
\captionsetup[table]{labelfont=it}
\usepackage{pifont}% http://ctan.org/pkg/pifont

\newcommand{\cmark}{\ding{51}}%
\newcommand{\xmark}{\ding{55}}%

\usepackage{hyperref}
\hypersetup{
  colorlinks=true,
  linkcolor=blue,
  filecolor=magenta,      
  urlcolor=cyan,
  pdftitle={Architettura del Software},
  pdfpagemode=FullScreen,
}

\usepackage{algorithm}
\usepackage{algpseudocode}

\newtheorem{nota}{Nota}

\usepackage[english]{babel}
\usepackage[
  backend=bibtex,
  style=numeric,
  sorting=ydnt
]{biblatex}
\addbibresource{quotes.bib}
\makeindex

\newcommand{\putimage}[4] {
  \begin{figure}[H]
    \centering
    \includegraphics[width={#4}\linewidth]{#1}
    \caption{#2}\label{#3}
  \end{figure}
}

\newcommand{\putsubimage}[5] {
  \begin{minipage}{{#4}\linewidth}
    \centering
    \includegraphics[width={#5}\linewidth]{#1}
    \caption{#2}\label{#3}
  \end{minipage}
}

\newcommand{\putimagecouple}[2] {
  \begin{figure}[!htb]
    \centering
    #1
    \hspace{0.5cm}
    #2
  \end{figure}
}

\newcommand{\putimagetriple}[3] {
  \begin{figure}[!htb]
    \centering
    #1
    #2
    #3
  \end{figure}
}

\newcommand{\putimagequadruple}[4] {
  \begin{figure}[!htb]
    \centering
    #1
    \hspace{0.5cm}
    #2
    \linebreak
    #3
    \hspace{0.5cm}
    #4
  \end{figure}
}

\begin{document}
  %\begin{titlepage}
  \noindent
  \begin{minipage}[t]{0.19\textwidth}
    \vspace{-4mm}{\includegraphics[scale=1.15]{logo_unimib.pdf}}
  \end{minipage}
  \begin{minipage}[t]{0.81\textwidth}
    {
      \setstretch{1.42}
      {\textsc{University of Milano Bicocca}} \\
      \textbf{School of Science} \\
      \textbf{Department of Informatics, System and Communication} \\
      \textbf{Master Degree in Computer Science} \\
      \par
    }
  \end{minipage}
  %\vspace{40mm}
  \begin{center}
    {\LARGE{
      \setstretch{1.2}
      \textbf{Riassunto di "Deployability"} \\
      \textbf{Architettura del Software}
      \par
    }}
  \end{center}
  %\vspace{50mm}
  %\vspace{15mm}
  \begin{center}
    \large{Refolli Francesco} \large{865955}
  \end{center}
  %\restoregeometry
  %\end{titlepage}

  %\tableofcontents
  \renewcommand{\baselinestretch}{1.5}

  \section{\textit{Deployment}}

  Il \textbf{deployment} nel settore dello sviluppo software \`e il processo di implementazione e distribuzione del software o di sue modifiche. Il processo assume il nome di \textit{continuous deployment} se \`e completamente automatizzato o di \textit{continuous delivery} se necessita dell'intervento umano per scatenare alcune azioni. Per velocizzare il processo si utilizza una sequenze di azioni e strumenti (\textit{pipeline}) che processano il software in vario modo per testarlo, verificare alcune caratteristiche o costruire artefatti utili alla distribuzione. La pipeline muove una versione di un modulo del software sequenzialmente in diversi ambienti:

  \begin{itemize}
    \item Un ambiente di \textit{development} dove avviene la scrittura di codice e dove vengono svolti tests di unit\`a sui suoi componenti.
    \item Un ambiente di \textit{intergration} dove viene testato insieme al resto del sistema.
    \item Un ambiente di \textit{staging} dove viene ulteriormente valutato in termini di sicurezza, performance o conformit\`a alle regolamentazioni.
    \item Un ambiente di \textit{production} dove il software \`e dispiegato per essere usato e monitorato.
  \end{itemize}

  Una pipeline ben costruita permette di avere cicli di breve durata, tracciabilit\`a completa in caso di errori (tramite artefatti o metadati) e idempotenza nelle azioni se ripetute tramite gli stessi input (cosa non scontata se esistono fattori ambientali non-deterministici come dipendenza da versioni "latest" o dati di contesto in file o basi di dati).

  Spesso emerge la necessit\`a di implementare meccanismi per includere o escludere features dalle istanze in modo dinamico (o statico) e di disporre di compatibilit\`a con le versioni precedenti. Queste sono scelte architetturali e possono avere un impatto sulla qualit\`a della soluzione e sulla sua \textit{deployability}.

  \subsection{\textit{Deployability}}

  Il termine \textit{deployability} indica che il deployment di un software \`e un processo dal prevedibile impegno in termini di risorse, tempo e costo. Inoltre in caso di problemi deve essere sempre possibile effettuare il rollback alla versione precedente per limitare i danni e riportare la versione problematica in sede di sviluppo.

  Si tratta chiaramente di una propriet\`a molto discrezionale, siccome si possono valutare numerosi aspetti: il \textbf{meccanismo} di \textit{raggiungimento} e \textit{aggiornamento} delle versioni \footnote{Include il medium, ex: DVD, USB, Internet, ...}, la \textbf{granularit\`a} di modifica \footnote{Pu\`o essere importante poter sostituire dei moduli senza portare down l'intero applicativo}, il tipo di packaging scelto \footnote{Syspackage, archivio, plugin, flatpack, container ...} e l'efficienza del processo. La valutazione di questi aspetti non \`e per forza univoca e in base alle circostanze determinate soluzioni possono essere pi\`u vantaggiose di altre \footnote{Ex: se i container sono comodi per dispiegare applicazioni headless, non si pu\`o dire lo stesso di applicazioni grafiche}.

  \section{Tattiche}

  \subsection{Gestione della Pipeline di Deployment}

  \paragraph{Scale rollouts}

  Restringere l'aggiornamento di versione ad un gruppo controllato di utenti e poi allargare gradualmente la platea di utenza servita dalla nuova versione permette di monitorare le modifiche minimizzando possibili i danni in caso di guasto.

  \paragraph{Roll back}

  In caso di malfunzionamenti \`e possibile ripristinare la versione precedente nelle istanze aggiornate o ricrearle direttamente.

  \paragraph{Script deployment commands}

  \`E assolutamente necessario automatizzare tutti i processi tediosi. Questo include anche le azioni di deployment, che possono andare dalla pacchettizzazione, al trasferminento e all'installazione delle nuove versioni di un software.

  \subsection{Gestione del Sistema in Deployment}

  \paragraph{Manage service interactions}

  Si intercede nelle richieste per mitigare le incompatibilit\`a dovute a cambiamenti di interfacce, dando la possibilit\`a di dispiegare multiple versioni di un servizio contemporaneamente e mascherarle dietro ad un load balancer senza rischiare inconsistenze.

  \paragraph{Package dependencies}

  \`E possibile pacchettizare il software in modo da far contenere alla distribuzione le sue dipendenze \footnote{Chiaramente questi pacchetti saranno pi\`u pesanti}. \`E l'approccio che seguono modelli come Flatpack, AppImage ... \cite{cichocki2023comparative}, tuttavia tecnologie come Nix \cite{dolstra2004nix} e Guix \cite{courtes2013functional} hanno introdotto la possibilit\`a di avere un sistema operativo che gestisce le dipendenze dei pacchetti in modo da isolare gli applicativi e mantenere diverse versioni delle stesse librerie.

  \paragraph{Feature toggles}

  Come detto precedentemente, in base al tipo di features che si vogliono introdurre, ai bisogni di personalizzazione \footnote{Caso delle applicazione multi-tenant} si pu\`o rendere necessario un meccanismo di selezione delle opzioni per una istanza \footnote{Eventualmente replicabile come file di configurazione o simili} in modo da ritardare o anticipare l'introduzione di una miglioria d'impatto per il cliente o per il sistema \footnote{Ex: una nuova versione del calcolo delle scadenze o dell'importazione dati}. Questo meccanismo pu\`o essere utile per implementare strategie di rollout incrementale.

  \section{Patterns}

  \printbibliography[title={Bibliografia}]
  \printindex
\end{document}
